\documentclass[a4paper,11pt]{scrartcl}
\usepackage[T1]{fontenc}
\usepackage[utf8]{inputenc}
\usepackage{lmodern}
\usepackage{ngerman}
\usepackage{graphicx}
\usepackage{xspace}
\usepackage{listings}
\usepackage{color}
\usepackage[hyphens]{url}
\usepackage{hyperref}
\usepackage{amssymb} % mathbb
\usepackage{amsmath}% http://ctan.org/pkg/amsmath

\newcommand{\zB}{\mbox{z.\,B.}\xspace}
\newcommand{\bspw}{\mbox{bspw.}\xspace}
\newcommand{\bzw}{\mbox{bzw.}\xspace}
\newcommand{\iAllg}{\mbox{i.\,Allg.}\xspace}
\newcommand{\ua}{\mbox{u.\,a.}\xspace}
\newcommand{\vs}{\mbox{vs.}\xspace}
\newcommand{\inkl}{\mbox{inkl.}\xspace}

\def\CC{{C\nolinebreak[4]\hspace{-.05em}\raisebox{.4ex}{\tiny\bf ++}}}
\def\GPP{{G\nolinebreak[4]\hspace{-.05em}\raisebox{.4ex}{\tiny\bf ++}}}
\setlength{\parindent}{0em} % Einrückung verhindern

\definecolor{mygreen}{rgb}{0,0.6,0}
\definecolor{mygray}{rgb}{0.5,0.5,0.5}
\definecolor{mymauve}{rgb}{0.58,0,0.82}

% \fontfamily{pzc}\selectfont
% \ttfamily

\lstset{ %
  backgroundcolor=\color{white},   % choose the background color; you must add \usepackage{color} or \usepackage{xcolor}; should come as last argument
  basicstyle=\footnotesize\fontfamily{bch}\selectfont,        
  % the size & type of the fonts that are used for the code
  breakatwhitespace=false,         % sets if automatic breaks should only happen at whitespace
  breaklines=true,                 % sets automatic line breaking
  captionpos=b,                    % sets the caption-position to bottom
  commentstyle=\color{mygreen},    % comment style
  deletekeywords={min,max},            % if you want to delete keywords from the given language
  escapeinside={\%*}{*)},          % if you want to add LaTeX within your code
  extendedchars=true,              % lets you use non-ASCII characters; for 8-bits encodings only, does not work with UTF-8
  frame=single,	                   % adds a frame around the code
  keepspaces=true,                 % keeps spaces in text, useful for keeping indentation of code (possibly needs columns=flexible)
  keywordstyle=\color{blue},       % keyword style
  language=C,                 % the language of the code
  morekeywords={},                 % if you want to add more keywords to the set
  numbers=left,                    % where to put the line-numbers; possible values are (none, left, right)
  numbersep=5pt,                   % how far the line-numbers are from the code
  numberstyle=\tiny\color{mygray}, % the style that is used for the line-numbers
  rulecolor=\color{black},         % if not set, the frame-color may be changed on line-breaks within not-black text (e.g. comments (green here))
  showspaces=false,                % show spaces everywhere adding particular underscores; it overrides 'showstringspaces'
  showstringspaces=false,          % underline spaces within strings only
  showtabs=false,                  % show tabs within strings adding particular underscores
  stepnumber=1,                    % the step between two line-numbers. If it's 1, each line will be numbered
  stringstyle=\color{mymauve},     % string literal style
  tabsize=2,	                   % sets default tabsize to 2 spaces
%   title=\lstname                   % show the filename of files included with \lstinputlisting; also try caption instead of title
}

%%%%%%%%%%%%%%%%%%%%%%%%%%%%%%%%%%%%%%%%%%%%%%%%%%%%%%%%%%%%%%%%%%%%%%%%%
%%%%%%%%%%%%%%%%%%%%%%%%%%%%%%%%%%%%%%%%%%%%%%%%%%%%%%%%%%%%%%%%%%%%%%%%%
  
\title{\includegraphics[width=0.6\textwidth]{bilder/tuc-logo-black.pdf}
\\Effiziente Implementierung von Matrix-Algorithmen für Multicore-Systeme
}
\subtitle{Praktikum Forschungsschwerpunkt Parallele und verteilte Systeme}
\author{Autor: Matthias Tietz}


\begin{document}

\maketitle \thispagestyle{empty}

\newpage
%%% Informationen/Leerseite %%%
\thispagestyle{empty}
~
\vfill
Technische Universität Chemnitz\\
Fakultät für Informatik\\
Professur Praktische Informatik\\
Praktikum Forschungsschwerpunkt Parallele und verteilte Systeme\\
Wintersemester 2016/2017\\

Effiziente Implementierung von Matrix-Algorithmen für Multicore-Systeme\\
Autor: Matthias Tietz\\
Matrikelnummer:~375681\\
Bachelor Informatik, 5.~Fachsemester

\newpage \tableofcontents
\newpage

\section{Einleitung}
Matrix-Algorithmen finden in einer Vielzahl verschiedener Bereiche Anwendung.
So werden \bspw in der linearen Algebra häufig Matrizen eingesetzt um linearere Gleichungssysteme oder Eigenwertprobleme zu lösen. Im Kontext linearer Abbildungen lassen sich geometrische Transformationen
durch Matrizenprodukte abbilden, was als Grundlage für die Computergrafik zur Realisierung von Koordinatentransformationenen dient.\newline

Der naive Standardalgorithmus zur Matrixmultiplikation 
$C\,= A \cdot B$ mit $A \in \mathbb{R}^{l \times m}$, $B \in \mathbb{R}^{m \times n}$,
$C \in \mathbb{R}^{l \times n}$ verfügt über eine kubische Laufzeit, für jedes Element $c$ der 
Ergebnismatrix $C$ müssen die Werte eines Zeilenvectors von $A$ mit den Werten eines Spaltenvectors
von $B$ schrittweise multipliziert und in $c$ aufsummiert werden. Die Ergebnismatrix $C$
hat die Dimensionen $l \times n$ und für jedes Element aus $C$ entsteht ein linearer Aufwand
$m \Rightarrow$ $\mathcal{O}(m \cdot l \cdot n)$.\newline

Der Algorithmus ist also einfach aufgebaut, es finden nur mathematisch grundlegende Operationen
(Multiplikation, Addition) statt, jedoch in einer großen Anzahl. 
Desweiteren werden die Werte aus
den Ausgangsmatrizen mehrfach verwendet, um \bspw die erste Zeile der Ergebnismatrix $C$ zu berechnen,
benötigen wir für jedes einzelne Element $c$ den gleichen Zeilenvector aus $A$, das gilt für die
Spaltenvectoren von $B$ analog.\newline

Aus den genannten Gründen ergibt sich das Potenzial, das Vorgehen bei der Matrixmultiplikation
zu verbessern. So könnte man die for()-Schleifen so umstrukturieren, dass man diese in kleinere
Iterationsblöcke aufteilt und damit vom Lokalitätsprinzip Gebrauch macht. Ein weiterer Ansatz wäre
ein paralleles Verfahren, um den Zeitaufwand zu minimieren, in dem mehrere Operationen in einem
Schritt ausgeführt werden.

\section{Implementierung}
Bei der Umsetzung der Matrix-Algorithmen ist in erster Linie die Performance-Verbesserung 
von Bedeutung, daher wurde bei der Implementierung kaum Wert auf eine besonders
gute Benutzerfreundlichkeit, umfassende Parameterprüfung oder Tests gelegt --
man geht davon aus, dass der Nutzer korrekte Eingabewerte verwendet.\newline

Zu Beginn der Implementierung hatte ich für die Matrizen einen Datentyp als
\texttt{struct} definiert, was sich aber als unnötig herausgestellt hat, der
Code sollte auf das Notwendige reduziert sein. Daher werden die einzelnen 
Matrizen einfach als float-Array umgesetzt. Zur Vereinfachung der verschiedenen
Verfahren wurde eine Einschränkung auf quadratische Matrizen ($n \times n$) 
und Parametergrößen für $n$ als Zweierpotenzen vorgenommen.\newline

Mit dem Ziel der Optimierungen, wird auch bewusst auf einen guten Coding-Stil verzichtet,
so werden in den performancerelevanten Bereichen unnötige Funktionsaufrufe oder Variablen vermieden, was einen schlechter lesbaren Code zur Folge hat. So wird \bspw anstatt einer
Hilfsfunktion zum Setzen eines Wertes in einer bestimmten Matrix

\lstinputlisting[language=C]{code/set.c}
direkt \texttt{result[(N * i) + j] = value;} verwendet.\newline

Die Ausgangsmatrizen $A$ und $B$
werden vor Benutzung in den verschiedenen Verfahren mit pseudo-zufälligen Werten initialisert
(\texttt{float$*$ createRandomizedMatrix\_f(int N);}), da die konkreten Werte nicht von großer Bedeutung sind, sollte dieses Vorgehen ausreichend sein. Die Ergebnismatrizen werden außerhalb des eigentlichen
Algorithmus mit Nullen initialisiert (\texttt{calloc()}). Das Repository \inkl einer kleinen
Beschreibung ist auf GitHub \cite{ghub} frei verfügbar.


\subsection{Standardalgorithmus}
Der Standardalgorithmus birgt keine großen Überraschungen und wurde in Anlehnung auf die 
Gleichung aus der Aufgabenstellung umgesetzt. In den beiden äußeren Schleifen 
wird über die Zeilen und Spalten der Ergebnismatrix \texttt{result} iteriert,
in der innersten Schleife wird dann die Produktsumme der Eingangsmatrizen (\texttt{a, b}) in der
Variable \texttt{calc} gebildet und der Wert anschließend gespeichert.
Einzige Optimierung ist hierbei der direkte Zugriff per Index auf die Matrizen.

\lstinputlisting[language=C]{code/std.c}

\subsection{Cache-optimiertes Verfahren}
Bei dem Cache-optimiertes Verfahren fand die Umsetzung mittels \emph{Loop tiling} statt,
wodurch die Ausführung der Schleifen effizienter wird. Das Loop tiling gliedert die 
Matrizen in kleinere Blöcke, auf denen die Matrixmultiplikation durchgeführt wird.
Dadurch verbleiben Daten, die bald wieder verwendet werden, im schnellen Hauptspeicher der 
CPU (Cache). \newpage

\lstinputlisting[language=C]{code/block.c}
%int optimizedMatrixMul_f(float* a, float* b, float* result, int N, int BS) {
Die Funktionssignatur besteht nun neben den bekannten Matrizen \texttt{a, b} und \texttt{result}
und der Dimension \texttt{N} nun noch zusätzlich aus einem Parameter \texttt{BS} für 
die Blockgröße, in die die Matrizen unterteilt werden sollen.\newline

In den äußeren drei Schleifen (Zeile 4-6) wird dafür gesorgt, dass der korrekte Offset eingehalten wird,
sodass anstatt über die komplette Dimension \texttt{N} der Matrizen zu iterieren, nur auf den
kleinen Blöcken der Größe \texttt{BS} gearbeitet wird. Daher gilt zu beachten, dass die Indexvariablen
\texttt{i,~j} und \texttt{k} nicht einfach inkrementiert werden wie bei herkömmlichen Schleifen,
sondern um den Offset-Wert \texttt{BS} erhöht werden. Daher auch die Forderung, dass 
die Parametergrößen als Zweierpotenz zu wählen sind, sodass $N \bmod BS = 0$, sich also die 
Matrizen restlos in Blöcke unterteilen lassen.\newline

Die inneren Schleifen führen dann 
für die einzelnen Blöcke eine reguläre Matrixmultiplikation aus.
% bezug ijk -> zyx (übergeben)

\subsection{Paralleles Verfahren}
\subsection{BLAS}


\section{Benchmark}
%bezug main ... wiederholungen, etc
Benutzung: Github, Parameter Kommandozeile etc.
korrekte Berechnung <-> Std.algor., funktion nearlyEqual

\section{Zusammenfassung}

\section{Quellen}
\begin{thebibliography}{9}

\bibitem{ghub} \url{https://github.com/matthiastz/FSP-Matrixmultiplikation}, März 2017

\end{thebibliography}


\end{document}
