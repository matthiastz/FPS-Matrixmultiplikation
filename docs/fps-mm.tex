\documentclass[a4paper,11pt]{scrartcl}
\usepackage[T1]{fontenc}
\usepackage[utf8]{inputenc}
\usepackage{lmodern}
\usepackage{ngerman}
\usepackage{graphicx}
\usepackage{xspace}
\usepackage{listings}
\usepackage{color}
\usepackage[hyphens]{url}
\usepackage{hyperref}
\usepackage{amssymb} % mathbb

\newcommand{\zB}{\mbox{z.\,B.}\xspace}
\newcommand{\bspw}{\mbox{bspw.}\xspace}
\newcommand{\bzw}{\mbox{bzw.}\xspace}
\newcommand{\iAllg}{\mbox{i.\,Allg.}\xspace}
\newcommand{\ua}{\mbox{u.\,a.}\xspace}
\newcommand{\vs}{\mbox{vs.}\xspace}

\def\CC{{C\nolinebreak[4]\hspace{-.05em}\raisebox{.4ex}{\tiny\bf ++}}}
\def\GPP{{G\nolinebreak[4]\hspace{-.05em}\raisebox{.4ex}{\tiny\bf ++}}}
\setlength{\parindent}{0em} % Einrückung verhindern

%%%%%%%%%%%%%%%%%%%%%%%%%%%%%%%%%%%%%%%%%%%%%%%%%%%%%%%%%%%%%%%%%%%%%%%%%
%%%%%%%%%%%%%%%%%%%%%%%%%%%%%%%%%%%%%%%%%%%%%%%%%%%%%%%%%%%%%%%%%%%%%%%%%
  
\title{\includegraphics[width=0.6\textwidth]{bilder/tuc-logo-black.pdf}
\\Effiziente Implementierung von Matrix-Algorithmen für Multicore-Systeme
}
\subtitle{Praktikum Forschungsschwerpunkt Parallele und verteilte Systeme}
\author{Autor: Matthias Tietz}


\begin{document}

\maketitle \thispagestyle{empty}

\newpage
%%% Informationen/Leerseite %%%
\thispagestyle{empty}
~
\vfill
Technische Universität Chemnitz\\
Fakultät für Informatik\\
Professur Praktische Informatik\\
Praktikum Forschungsschwerpunkt Parallele und verteilte Systeme\\
Wintersemester 2016/2017\\

Effiziente Implementierung von Matrix-Algorithmen für Multicore-Systeme\\
Autor: Matthias Tietz\\
Matrikelnummer:~375681\\
Bachelor Informatik, 5.~Fachsemester

\newpage \tableofcontents
\newpage

\section{Einleitung}
%Hinführung zum Thema, MM, mögliche Optimierung
Matrix-Algorithmen finden in einer Vielzahl verschiedener Bereiche Anwendung.
So werden \bspw in der linearen Algebra häufig Matrizen eingesetzt um linearere Gleichungssysteme oder Eigenwertprobleme zu lösen. Im Kontext linearer Abbildungen lassen sich geometrische Transformationen
durch Matrizenprodukte abbilden, was als Grundlage für die Computergrafik zur Realisierung von Koordinatentransformationenen dient.\newline

Der naive Standardalgorithmus zur Matrixmultiplikation 
$C\,= A \cdot B$ mit $A \in \mathbb{R}^{l \times m}$, $B \in \mathbb{R}^{m \times n}$,
$C \in \mathbb{R}^{l \times n}$ verfügt über eine kubische Laufzeit, für jedes Element $c$ der 
Ergebnismatrix $C$ müssen die Werte eines Zeilenvectors von $A$ mit den Werten eines Spaltenvectors
von $B$ schrittweise multipliziert und in $c$ aufsummiert werden. Die Ergebnismatrix $C$
hat die Dimensionen $l \times n$ und für jedes Element aus $C$ entsteht ein linearer Aufwand
$m \Rightarrow$ $\mathcal{O}(m \cdot l \cdot n)$.\newline

Der Algorithmus ist also einfach aufgebaut, es finden nur mathematisch grundlegende Operationen
(Multiplikation, Addition) statt, jedoch in einer großen Anzahl. 
Desweiteren werden die Werte aus
den Ausgangsmatrizen mehrfach verwendet, um \bspw die erste Zeile der Ergebnismatrix $C$ zu berechnen,
benötigen wir für jedes einzelne Element $c$ den gleichen Zeilenvector aus $A$, das gilt für die
Spaltenvectoren von $B$ analog.


\section{Implementierung}
Indexzugriff, wenig unnötige Funktionsaufrufe, schlechter lesbar
Vor/Nachteile
\subsection{Standardalgorithmus}
\subsection{Cache-optimiert}
\subsection{paralleles Verfahren}
\subsection{BLAS}


\section{Benchmark}

\section{Zusammenfassung}

\section{Quellen}
Stil: siehe HS


\end{document}
